%% example-assignment-english.tex
%% Copyright 2021 Lukas Schmelzeisen
%
% This work may be distributed and/or modified under the
% conditions of the LaTeX Project Public License, either version 1.3
% of this license or (at your option) any later version.
% The latest version of this license is in
%   http://www.latex-project.org/lppl.txt
% and version 1.3 or later is part of all distributions of LaTeX
% version 2005/12/01 or later.
%
% This work has the LPPL maintenance status `maintained'.
% 
% The current maintainer of this work is Lukas Schmelzeisen.
%
% This work consists of the files acAssignment.cls and acAssignment.tex and
% bundled example files.


% Enable warnings about problematic code
\RequirePackage[l2tabu, orthodox]{nag}

\documentclass[solution]{acAssignment}

% Parse this file as UTF-8 (included in LaTeX by default since 2018 but included
% here for backwards-compatibility). If you use something else, change this.
\usepackage[utf8]{inputenc}

\acIfSolution{
    \usepackage{tikz, forest}
}{}

\acCourse[https://ilias3.uni-stuttgart.de/goto_Uni_Stuttgart_crs_2120363.html]
    {Introduction to Artificial Intelligence}
\acSemester{WS~20/21}
\acAssignmentNumber{3}
\acAssignmentTitle{Probabilistic Logic and First-order Logic}

\acAuthor[https://www.ipvs.uni-stuttgart.de/institute/team/Potyka-00001/]%
    {Dr. Nico Potyka}{nico.potyka@ipvs.uni-stuttgart.de}
\acAuthor[https://www.ipvs.uni-stuttgart.de/institute/team/Kumar-00006/]%
    {Dr. Chandan Kumar}{chandan.kumar@ipvs.uni-stuttgart.de}
\acAuthor[https://www.ipvs.uni-stuttgart.de/institute/team/Kraemer/]%
    {Teresa Krämer}{teresa.krämer@ipvs.uni-stuttgart.de}
\acAuthor[https://www.ipvs.uni-stuttgart.de/institute/team/Iurshina/]%
    {Anastasiia Iurshina}{anastasiia.iurshina@ipvs.uni-stuttgart.de}
\acAuthor[https://www.ipvs.uni-stuttgart.de/institute/team/Schmelzeisen/]%
    {Lukas Schmelzeisen}{lukas.schmelzeisen@ipvs.uni-stuttgart.de}
\acAuthor[https://www.ipvs.uni-stuttgart.de/institute/team/Sengupta-00002/]%
    {Korok Sengupta}{korok.sengupta@ipvs.uni-stuttgart.de}

\begin{document}
\maketitle

This assignment sheet consists of \acNumPages~pages with the following \acNumTasks~tasks:

\acListOfTasks

Submit your solution in ILIAS as a single PDF file.%
\footnote{Your drawing software probably allows to export as PDF.
    An alternative option is to use a PDF printer.
    If you create multiple PDF files, use a merging tool (like \href{https://github.com/pdfarranger/pdfarranger}{pdfarranger}) to combine the PDFs into a single file.}
Make sure to list your full name and immatriculation number at the start of the file.
Optionally, you can \emph{additionally} upload source files (e.g. PPTX files).
Remember to fill out the exercise slot and exercise presentation polls linked in ILIAS.
If you have any questions, feel free to ask them in the excercise forum in ILIAS.

\textbf{Submission is open until Saturday, 5~December 2020, 23:59.}

% ------------------------------------------------------------------------------
\section{Practice applying theory and analytical thinking}

\begin{enumerate}
    \item \acTask[2 and 3 points]{5}:
        Use the definition of logical equivalence to prove the following equivalences analogous to the proofs of commutativity and implication elimination on the slides:
        \begin{enumerate}
            \item Double negation
            
                \begin{acSolution}
                    \begin{align}
                        \mathrm{Mod}(\lnot \lnot F)
                            &= \{I \in \mathrm{Int}(\Sigma) \mid I ~\mathrm{satisfies}~ \lnot \lnot F\} \\
                            &= \{I \in \mathrm{Int}(\Sigma) \mid I ~\mathrm{falsifies}~ \lnot F\} \\
                            &= \{I \in \mathrm{Int}(\Sigma) \mid I ~\mathrm{satisfies}~ F\} \\
                            &= \mathrm{Mod}(F)
                    \end{align}
                \end{acSolution}
            
            \item Contraposition
            
                \begin{acSolution}
                    \begin{align}
                        \mathrm{Mod}(A \to B)
                            &= \{I \in \mathrm{Int}(\Sigma) \mid I ~\mathrm{falsifies}~ A ~\mathrm{or}~ I ~\mathrm{satisfies}~ B\} \\
                            &= \{I \in \mathrm{Int}(\Sigma) \mid I ~\mathrm{falsifies}~ \lnot B ~\mathrm{or}~ I ~\mathrm{satisfies}~ \lnot B\} \\
                            &= \mathrm{Mod}(\lnot B \to \lnot A)
                    \end{align}
                \end{acSolution}
        \end{enumerate}
        
    \item \acTask{5}:
        Use the deduction theorem, biimplication elimination and the relationship between logical equivalence and logical consequence to explain the following fact: $F \equiv G$ if and only if $F \leftrightarrow G$ is a tautology.
        
        \begin{acSolution}
            We have $F \equiv G$\\
            iff $F \models G$ and $G \models F$ (follows immediately from definitions, see lecture)\\
            iff $F \to G$ is a tautology and $G \to F$ is a tautology (deduction theorem)\\
            iff $F \leftrightarrow G$ is a tautology (biimplication elimination).
        \end{acSolution}
        
    \item The \enquote{contraposition} of a statement \enquote{If A, then B} is the statement \enquote{If not B, then not A}.
    
        \acTask[1 and 2 Points]{3}: Form the contrapositions of the following statements:
        
        \begin{enumerate}
            \item If $x$ is a local minimum, then the gradient at $x$ is zero.
            
                \begin{acSolution}
                    If the gradient at $x$ is not $0$, then $x$ is not a local minimum.
                \end{acSolution}
            
            \item If all costs are bounded from below by a positive constant and the heuristic is admissible, then A*~search stops with an optimal solution.
            
                \begin{acSolution}
                    If A*~search does not stop with an optimal solution, then not all costs are bounded from below by a positive constant or the heuristic is not admissible.
                \end{acSolution}
        \end{enumerate}
        
    \acIfSolution{\clearpage}{}
    \item Consider the knowledge base $\mathcal{K} = \{A \to \lnot B, B \land C \to A, \lnot A \to C, \lnot C \to \lnot A\}$.
    
        \begin{enumerate}
            \item \acTask{4}: Represent $\mathcal{K}$ by a formula in CNF.
            
                \begin{acSolution}
                    We first rewrite the knowledge base equivalently as the conjunction of its elements:
                    
                    \begin{equation}
                        (A \to \lnot B) \land (B \land C \to A) \land (\lnot A \to C) \land (\lnot C \to \lnot A).
                    \end{equation}
                    
                    By applying the implication elimination, De Morgan's laws and double negation, the implications can be turned into disjunctions:
                    
                    \begin{equation}
                        (\lnot A \lor \lnot B) \land (\lnot B \lor \lnot C \lor A) \land (A \lor C) \land (C \lor \lnot A).
                    \end{equation}
                    
                    The resulting formula is a conjuncation of disjunctions and thus in CNF.
                \end{acSolution}
            
            \item \acTask{2}: Represent $\mathcal{K}$ in clause form.
            
                \begin{acSolution}
                    To represent the formula in clause form, we simply replace every disjunction with a clause:
                    
                    \begin{equation}
                        \big\{ \{\lnot A, \lnot B\}, \{\lnot B, \lnot C, A\}, \{A, C\}, \{C, \lnot A\} \big\}.
                    \end{equation}
                \end{acSolution}
            
            \item \acTask{2}: Which clauses are definite clauses?
            
                \begin{acSolution}
                    The clauses $\{\lnot B, \lnot C, A\}$ and $\{C, \lnot A\}$ are definite clauses.
                \end{acSolution}
            
            \item \acTask{2}: Which clauses are goal clauses?
            
                \begin{acSolution}
                    The clause $\{\lnot A, \lnot B\}$ is a goal clause.
                \end{acSolution}
            
            \item \acTask{2}: Which clauses are Krom clauses?
            
                \begin{acSolution}
                    The clauses $\{\lnot A, \lnot B\}$, $\{A, C\}$, and $\{C, \lnot A\}$ are Krom clauses.
                \end{acSolution}
        \end{enumerate}
\end{enumerate}

% ------------------------------------------------------------------------------
\section{Resolution with Clause Graphs}

\begin{enumerate}
    \item Consider the following clauses:
    
        \begin{itemize}
           \item $\{B\}$,
           \item $\{A, B\}$,
           \item $\{C, F\}$,
           \item $\{\lnot D, F\}$,
           \item $\{\lnot C, D\}$,
           \item $\{\lnot C, E\}$,
           \item $\{\lnot E, F\}$,
           \item $\{\lnot A, \lnot B, C\}$.
        \end{itemize}
        
        \acTask{9}:
        Use a presentation software like Powerpoint or Libre Office Impress to draw the clause graph.
    
    \item \acTask{3}:
        Perform a resolution step that allows you to apply the isolation rule.
        Draw the new graph (just copy the previous one and adapt it) and explain what you did.
    
    \item \acTask{3}:
        Perform a resolution step that allows you to delete a tautology.
        Draw the new graph (just copy the previous one and adapt it) and explain what you did.
\end{enumerate}

\begin{acSolution}
    Solution missing, as I did not want to take the time to typeset it in TikZ.
\end{acSolution}

% ------------------------------------------------------------------------------
\section{Resolution with Restrictions}

\begin{enumerate}
    \item Consider the following clauses:
        
        \begin{itemize}
            \item $\{B\}$,
            \item $\{\lnot D\}$,
            \item $\{\lnot C, D\}$,
            \item $\{\lnot A, \lnot B, C\}$.
        \end{itemize}
        
        \acTask{9}:
        Use resolution to prove by contradiction that the clauses entail $\lnot A$.
        Use a presentation software like Powerpoint or Libre Office Impress to draw the derivation of the empty clause similar to what you have seen on the slides.
        
        \begin{acSolution}
            For proof by contradiction, add $\lnot \lnot A \equiv A$ to the knowledge base.
            
            \begin{center}
                \begin{forest}
                    for tree={
                        grow'=90,
                        parent anchor=north,
                        math content,
                        before typesetting nodes={
                            if level=0{}{
                                if content={}
                                    {shape=coordinate}
                                    {content/.wrap value={\{#1\}}}}}}
                    [{\emptyset}
                        [
                            [
                                [
                                    [{A}]]]]
                        [{\lnot A}
                            [
                                [
                                    [{B}]]]
                            [{\lnot A, \lnot B}
                                [{\lnot C}
                                    [{\lnot D}]
                                    [{\lnot C, D}]]
                                [
                                    [{\lnot A, \lnot B, C}]]]]]
                \end{forest}
            \end{center}
        \end{acSolution}
    
    \item \acTask{3}:
        Is your proof a resolution proof?
        Explain why or why not?
        
        \begin{acSolution}
            The proof is a linear resolution proof because every center clause is derived from the last center clause and every side clause is an input clause.
        \end{acSolution}
        
    \item \acTask{3}:
        Is your proof an input resolution proof? Explain why or why not.
        
        \begin{acSolution}
            The proof is also an input resolution proof because every resolution step involves an input clause.
        \end{acSolution}
\end{enumerate}

% ------------------------------------------------------------------------------
\section{First-order Basics}

Consider a first-order signature with function symbols $\{\mathrm{alice, \mathrm{bob}, \mathrm{charles}, \mathrm{donald}, \mathrm{best\_friend}}\}$, where $\mathrm{alice}$, $\mathrm{bob}$, $\mathrm{charles}$, and $\mathrm{donald}$ are constants and $\mathrm{best\_friend}$ is a function symbol with arity $1$.
It contains the predicate symbols $\{\mathrm{rich}, \mathrm{knows}, \mathrm{equal}\}$, where $\mathrm{rich}$ is an unary and $\mathrm{knows}$ and $\mathrm{equals}$ are binary predicate symbols.
Consider the interpretation $\mathcal{I}$ with

\begin{itemize}
    \item $\mathcal{U}^\mathcal{I} = \{\mathrm{AliceCarroll}, \mathrm{BobBurns}, \mathrm{DonaldDuck}\}$,
    \item $\mathrm{alice}^\mathcal{I} = \mathrm{AliceCarroll}$,
    \item $\mathrm{bob}^\mathcal{I} = \mathrm{BobBurns}$,
    \item $\mathrm{charles}^\mathcal{I} = \mathrm{BobBurns}$,
    \item $\mathrm{donald}^\mathcal{I} = \mathrm{DonaldDuck}$,
    \item $\mathrm{best\_friend}^\mathcal{I}(\mathrm{AliceCarroll}) = \mathrm{DonaldDuck}$,
    \item $\mathrm{best\_friend}^\mathcal{I}(\mathrm{DonaldDuck}) = \mathrm{AliceCarroll}$,
    \item $\mathrm{best\_friend}^\mathcal{I}(\mathrm{BobBurns}) = \mathrm{BobBurns}$,
    \item $\mathrm{rich}^\mathcal{I} = \{\mathrm{DonaldDuck}\}$,
    \item $\mathrm{knows}^\mathcal{I} = \{(\mathrm{AliceCarroll}, \mathrm{DonaldDuck}), (\mathrm{DonaldDuck}, \mathrm{AliceCarroll})\}$,
    \item $\mathrm{equals}^\mathcal{I} = \{(e, e) \mid e \in \mathcal{U}^\mathcal{I}\}$.
\end{itemize}
    
\vspace{0.4cm}
\begin{enumerate}
    \item \acTask[1 Point each]{5}:
        Which of the following formulas is satisfied/falsified by $\mathcal{I}$?
        (A short answer is sufficient, but you should be able to explain it in detail in the live exercise.)
        
        \begin{itemize}
            \item $\mathrm{equal}(\mathrm{best\_friend}(\mathrm{charles}), \mathrm{charles})$
            
                \begin{acSolution}
                    We have $\mathcal{I} \models \mathrm{equal}(\mathrm{best\_friend}(\mathrm{charles}), \mathrm{charles})$ iff
                    
                    \begin{equation}
                        (\mathrm{best\_friend}^\mathcal{I}(\mathrm{charles}^\mathcal{I}), \mathrm{charles}^\mathcal{I}) \in \mathrm{equals}^\mathcal{I}.
                    \end{equation}
                    
                    We have
                    
                    \begin{itemize}
                        \item $\mathrm{charles}^\mathcal{I} = \mathrm{BobBurns}$,
                        \item $\mathrm{best\_friend}^\mathcal{I}(\mathrm{BobBurns}) = \mathrm{BobBurns}$, and
                        \item $(\mathrm{BobBurns}, \mathrm{BobBurns}) \in \mathrm{equals}^\mathcal{I}$.
                    \end{itemize}
                    
                    Hence, $\mathcal{I} \models \mathrm{equal}(\mathrm{best\_friend}(\mathrm{charles}), \mathrm{charles})$.
                \end{acSolution}
            \item $\mathrm{equal}(\mathrm{best\_friend}(\mathrm{bob}), \mathrm{charles})$
            
                \begin{acSolution}
                    Similarly, we have $\mathcal{I} \models \mathrm{equal}(\mathrm{best\_friend}(\mathrm{bob}), \mathrm{charles})$.
                \end{acSolution}
            
            \item $\mathrm{equal}(\mathrm{best\_friend}(\mathrm{alice}), \mathrm{bob})$
            
                \begin{acSolution}
                    $\mathcal{I}$ falsifies $\mathrm{equal}(\mathrm{best\_friend}(\mathrm{alice}), \mathrm{bob})$ because $\mathrm{bob}^\mathcal{I} = \mathrm{BobBurns}$ while $\mathrm{best\_friend}^\mathcal{I}(\mathrm{alice}^\mathcal{I}) = \mathrm{best\_friend}^\mathcal{I}(\mathrm{AliceCarrol}) = \mathrm{DonaldDuck}$ and $(\mathrm{DonaldDuck}, \mathrm{BobBurns}) \notin \mathrm{equal}^\mathcal{I}$.
                \end{acSolution}
                
            \item $\forall x \, \forall y \, (\mathrm{knows}(x, y) \to \mathrm{knows}(y, x))$
            
                \begin{acSolution}
                    The premise $\mathrm{knows}(x, y)$ of the implication is only satisfied for $\mathrm{AliceCarroll}$ and $\mathrm{DonaldDuck}$ and, in this case, $\mathrm{knows}(y, x)$ is true as well.
                    Therefore, the formula is satisfied.
                \end{acSolution}
            
            \item $\exists x \,\big(\lnot \mathrm{knows}(x, \mathrm{best\_friend}(x))\big)$
            
                \begin{acSolution}
                    The literal is satisfied for $\mathrm{BobBurns}$, so the existential statement is satisfied and $\mathcal{I}$ satisfies the formula.
                \end{acSolution}
        \end{itemize}
        
    \item \acTask{5}:
        Define another interpretation with at least 4 elements in the universe and interpret every predicate symbol by a non-empty set.
        
        \begin{acSolution}
            \begin{itemize}
                \item $\mathcal{U}^\mathcal{I} = \{a, b, c, d\}$
                \item $\mathrm{alice}^\mathcal{I} = a$,
                \item $\mathrm{bob}^\mathcal{I} = b$,
                \item $\mathrm{charles}^\mathcal{I} = c$,
                \item $\mathrm{donald}^\mathcal{I} = d$,
                \item $\mathrm{best\_friend}^\mathcal{I}(a) = a$,
                \item $\mathrm{best\_friend}^\mathcal{I}(b) = b$,
                \item $\mathrm{best\_friend}^\mathcal{I}(c) = c$,
                \item $\mathrm{best\_friend}^\mathcal{I}(d) = d$,
                \item $\mathrm{rich}^\mathcal{I} = \{x \mid x \in \mathcal{U}^\mathcal{I}\}$,
                \item $\mathrm{knows}^\mathcal{I} = \{(x,y) \mid (x, y) \in \mathcal{U}^\mathcal{I} \times \mathcal{U}^\mathcal{I}\}$,
                \item $\mathrm{equals}^\mathcal{I} = \{(x, x) \mid x \in \mathcal{U}^\mathcal{I}\}$.
            \end{itemize}
        \end{acSolution}
\end{enumerate}

\end{document}