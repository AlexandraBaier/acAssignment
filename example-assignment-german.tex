%% example-assignment-german.tex
%% Copyright 2021 Lukas Schmelzeisen
%
% This work may be distributed and/or modified under the
% conditions of the LaTeX Project Public License, either version 1.3
% of this license or (at your option) any later version.
% The latest version of this license is in
%   http://www.latex-project.org/lppl.txt
% and version 1.3 or later is part of all distributions of LaTeX
% version 2005/12/01 or later.
%
% This work has the LPPL maintenance status `maintained'.
% 
% The current maintainer of this work is Lukas Schmelzeisen.
%
% This work consists of the files acAssignment.cls and acAssignment.tex and
% bundled example files.

% Enable warnings about problematic code
\RequirePackage[l2tabu, orthodox]{nag}

\documentclass[german, solution]{acAssignment}

\acCourse[https://west.uni-koblenz.de/studying/ws1920/algorithmen-und-datenstrukturen]
    {Algorithmen und Datenstrukturen}
\acSemester{WS~19/20}
\acAssignmentNumber{3}
\acAssignmentTitle{O-Notation}

\acAuthor[https://west.uni-koblenz.de/de/about-us/team/matthias-thimm]
    {PD Matthias Thimm}{thimm@uni-koblenz.de}
\acAuthor[https://www.ipvs.uni-stuttgart.de/de/institut/team/Schmelzeisen/]
    {Lukas Schmelzeisen}{lukas.schmelzeisen@ipvs.uni-stuttgart.de}

\usepackage{array}

\begin{document}
\maketitle

Dieses Aufgabenblatt besteht aus \acNumPages~Seiten mit den folgenden \acNumTasks~Aufgaben:

\acListOfTasks

Bitte lesen Sie vor Abgabe die \hyperref[sec:submission-notes]{Hinweise zur Abgabe} am Ende des Aufgabenblatts.

\textbf{Die Abgabe ist bis Freitag, 22.~November 2021, 23:59 möglich.}


% ------------------------------------------------------------------------------
\section{Wachstumsverhältnisse zwischen Funktionen}

Die folgende Tabelle ist so zu verstehen: wenn die Funktion $f(n)$ in der
Spalte $x$ und die Funktion $g(n)$ in der Zeile $y$ stehen, so soll in die Zelle
$(x,y)$ das Zeichen $\mathcal{A} \in \{\mathcal{O}, \Omega, \Theta\}$ eingetragen
werden, wenn $f(n) \in \mathcal{A}(g(n))$ gilt.
($\Theta$ impliziert hierbei $\mathcal{O}$ und $\Omega$.)

\acTask[1 Punkte pro Zeile; pro Fehler 0.5 Punkte Abzug; mindestens 0 Punkte pro Zeile]{4}:
Füllen Sie die fehlenden Zellen in der Tabelle aus.

\vspace{0.4cm}

\begin{center}
    \newcommand*\OMI{\acInlineSolution{$\mathcal{O}$}}
    \newcommand*\OME{\acInlineSolution{$\Omega$}}
    \newcommand*\THE{\acInlineSolution{$\Theta$}}
    \let\oldarraystretch\arraystretch
    \renewcommand*{\arraystretch}{1.5}
    % See https://tex.stackexchange.com/a/12712/75225
    \newcolumntype{C}[1]{>{\centering\let\newline\\\arraybackslash\hspace{0pt}}m{#1}}
    \begin{tabular}{|c|*{6}{C{1.5cm}|}}
        \cline{2-7}
        \multicolumn{1}{c|}{} & $1000$ & $\lceil \frac{n}{2} \rceil$ & $\lceil \mathrm{log}(n) \rceil$ & $\lceil \sqrt{n} \rceil$ & $\lceil \frac{4^n}{4} \rceil$ & $5 n^2$ \\
        \hline
        $13 n$ & $\mathcal{O}$ & $\Theta$ & \OMI & \OMI & \OME & \OME \\
        \hline
        $\lceil \frac{\mathrm{log}(n)}{2} \rceil$ & \OMI & \OME & \THE & \OME & \OME & \OME \\
        \hline
        $3 n^2 + 2$ & \OMI & \OMI & \OMI & \OMI & \OME & \THE \\
        \hline
        $\lceil e^n \rceil$ & \OMI & \OMI & \OMI & \OMI & \OME & \OMI \\
        \hline
    \end{tabular}
    \let\arraystretch\oldarraystretch
\end{center}

\vspace{0.4cm}

\acNote:
Die Gaussklammern $\lceil \,\cdot\, \rceil$ runden ihr Argument immer zur nächst höheren ganzen Zahl auf.
Zum Beispiel gilt $\lceil 2.1 \rceil = 3$.


% ------------------------------------------------------------------------------
\section{Beziehungen zwischen Aufwandsklassen}

Seien $f$ und $g$ beliebige Funktionen mit $f : \mathbb{N} \to \mathbb{N}$ und $g : \mathbb{N} \to \mathbb{N}$.

\acTask[1~Punkt pro Aussage]{5}:
Zeigen oder widerlegen Sie die folgenden Aussagen:

\begin{enumerate}
    \item $f(n) + g(n) \in \Theta(\max(f(n), g(n)))$
    
        \begin{acSolution}
            In der Vorlesung wurde $\Theta(f(n)) = \mathcal{O}(f(n)) \cap \Omega(f(n)$ gezeigt.
            Damit können wir zu $f(n) + g(n) \in \Theta(\max(f(n), g(n))) = \mathcal{O}(\max(f(n), g(n))) \cap \Omega(\max(f(n), g(n))$ umformen.
            Damit dies gilt müssen folgende beiden Aussagen gelten:
            \begin{enumerate}[label=(\arabic*)]
            \item\label{statement1} $f(n) + g(n) \in \mathcal{O}(\max(f(n), g(n)))$
            \item\label{statement2} $f(n) + g(n) \in \Omega(\max(f(n), g(n)))$
            \end{enumerate}
            Aussage \ref{statement1} wurde in der Vorlesung gezeigt.
            Aussage \ref{statement2} gilt, da beispielsweise für $c=1$ stets $f(n) + g(n) \geq c \cdot \max(f(n), g(n))$ gilt, da aus $f:\mathbb{N} \to \mathbb{N}$ und $g:\mathbb{N} \to \mathbb{N}$ folgt, dass $f(n) > 0$ und $g(n) > 0$.
        \end{acSolution}
    
    \item $f(n) \in \mathcal{O}(g(n)) \implies g(n) \in \mathcal{O}(f(n))$
    
        \begin{acSolution}
            Gilt nicht, da $0 \in \mathcal{O}(1)$ aber nicht $1 \in \mathcal{O}(0)$, weil für alle $c \in \mathbb{R}^{>0}$ stets $0 \leq c \cdot 1$ gilt, aber es kein $c \in \mathbb{R}^{>0}$ gibt für das $1 \leq c \cdot 0$ gilt.
        \end{acSolution}
    
    \item $f(n) \in \mathcal{O}(g(n)) \implies 2^{f(n)} \in \mathcal{O}(2^{g(n)})$
    
        \begin{acSolution}
            Gilt nicht, da $2 \, n \in \mathcal{O}(n)$ aber nicht $2^{2 \, n} \in \mathcal{O}(2^n)$, weil beispielsweise mit $c = 2$ stets $2 \, n \leq c \cdot n$ gilt, aber es kein fixes $c \in \mathbb{R}^{>0}$ gibt für das gilt
            \begin{align*}
                & 2^{2 \, n} \leq c \cdot 2^n \\
                \iff & 2^n \cdot 2^n \leq c \cdot 2^n \\
                \iff & 2^n \leq c\\[-1cm]
            \end{align*}
        \end{acSolution}
    
    \item $f(n) \in \mathcal{O}(g(n)) \lor g(n) \in \mathcal{O}(f(n))$
    
        \begin{acSolution}
            Gilt nicht, beispielsweise gilt $(-1)^n \notin \mathcal{O}(-(-1)^n)$ und $-(-1)^n \notin \mathcal{O}((-1)^n)$, weil es kein fixes $c \in \mathbb{R}^{>0}$ gibt für das $(-1)^n \leq c \cdot -(-1)^n$ oder $-(-1)^n \leq c \cdot (-1)^n$ gilt.
            
            Eine anderes Beispiel wären $\mathrm{sin}(x)$ und $\mathrm{cos}(x)$, aber zu Beginn der Aufgaben wurden $f : \mathbb{N} \to \mathbb{N}$ gefordert und die trigonometrischen Funktionen sind in der Regeln nicht nur auf den natürlichen Zahlen definiert (kein Abzug hierfür).
        \end{acSolution}
    
    \item $\mathcal{O}(f(n)) \cap \Omega(f(n)) = \emptyset$
    
        \begin{acSolution}
            Gilt nicht, denn beispielsweise gilt $n \in \mathcal{O}(n) \cap \Omega(n)$, weil $n \in \mathcal{O}(n)$ (weil beispielsweise mit $c=1$ stets $n \leq c \cdot n$ gilt) und $n \in \Omega(n)$ (weil beispielsweise mit $c=1$ stets $n \geq c \cdot n$ gilt).
            
            Alternativ: In der Vorlesung wurde $\Theta(n) = \mathcal{O}(n) \cap \Omega(n)$ gezeigt und $\Theta(n) \neq \emptyset$ weil $n \in \Theta(n)$ (weil beispielsweise mit $c_1 = c_2 = 1$ stets $c_1 \cdot n \geq n \geq c_2 \cdot n$ gilt).
        \end{acSolution}
\end{enumerate}


% ------------------------------------------------------------------------------
\section{Aufwandsanalyse von Algorithmen}

Gegeben sind einige iterative und rekursive Algorithmen:

\NewEnviron{myAlgTable}[1]{%
    \textbf{Algorithmus #1}\par%
        \begin{tabular}{p{2.25cm}p{11.75cm}}%
            \toprule%
            \BODY%
            \bottomrule%
        \end{tabular}}
\newcommand*\myAlgTableRow[2]{%
    \ifstrempty{#1}{}{%
        $\mathcal{O}\,(\acIfSolution{\,\textcolor{acSolution}{#1}}{}\hfill)$}%
    &\texttt{#2}\\}
    
\let\oldarraystretch\arraystretch
\renewcommand*{\arraystretch}{1}
\begin{itemize}[itemsep=0.4cm]
    \item
        \begin{myAlgTable}{1}
            \myAlgTableRow{}{\textbf{void} alg1(\textbf{int} n) \{}
            \myAlgTableRow{1}{\ \ \textbf{int} j = n * n;}
            \myAlgTableRow{1}{\ \ \textbf{int} p = 0;}
            \myAlgTableRow{n^2}{\ \ \textbf{while} (j > 0) \{}
            \myAlgTableRow{1}{\ \ \ \ \textbf{int} i = 1;}
            \myAlgTableRow{\log\,n}{\ \ \ \ \textbf{while} (i < n) \{}
            \myAlgTableRow{1}{\ \ \ \ \ \ p = 2 * j;}
            \myAlgTableRow{1}{\ \ \ \ \ \ i = i * 2;}
            \myAlgTableRow{}{\ \ \ \ \}}
            \myAlgTableRow{1}{\ \ \ \ j = j - 1;}
            \myAlgTableRow{}{\ \ \}}
            \myAlgTableRow{}{\}}
        \end{myAlgTable}
        
    \item
        \begin{myAlgTable}{2}
            \myAlgTableRow{}{\textbf{void} alg2(\textbf{int} n) \{}
            \myAlgTableRow{1}{\ \ \textbf{int} m = 0;}
            \myAlgTableRow{\infty}{\ \ \textbf{for} (\textbf{int} i = 0; i <= n; i = i * 2) \{}
            \myAlgTableRow{1}{\ \ \ \ \textbf{int} j = 0;}
            \myAlgTableRow{i}{\ \ \ \ \textbf{while} (j < i) \{}
            \myAlgTableRow{1}{\ \ \ \ \ \ m = m + 1;}
            \myAlgTableRow{1}{\ \ \ \ \ \ j = j + 1;}
            \myAlgTableRow{}{\ \ \ \ \}}
            \myAlgTableRow{}{\ \ \}}
            \myAlgTableRow{}{\}}
        \end{myAlgTable}
        
    \item
        \begin{myAlgTable}{3}
            \myAlgTableRow{}{\textbf{void} alg3(\textbf{int} n) \{}
            \myAlgTableRow{1}{\ \ \textbf{int} i = 2;}
            \myAlgTableRow{n}{\ \ \textbf{while} (i <= n) \{}
            \myAlgTableRow{1}{\ \ \ \ \textbf{if} (!f(i))}
            \myAlgTableRow{\log\,n}{\ \ \ \ \ \ \textbf{for} (\textbf{int} j = 1; j <= n; j = j * 2)}
            \myAlgTableRow{j}{\ \ \ \ \ \ \ \ g(j);}
            \myAlgTableRow{1}{\ \ \ \ i = i + 1;}
            \myAlgTableRow{}{\ \ \}}
            \myAlgTableRow{1}{\ \ i = 0;}
            \myAlgTableRow{n}{\ \ \textbf{while} (i <= n) \{}
            \myAlgTableRow{1}{\ \ \ \ h(i);}
            \myAlgTableRow{1}{\ \ \ \ i = i + 1;}
            \myAlgTableRow{}{\ \ \}}
            \myAlgTableRow{}{\}}
        \end{myAlgTable}
        
        Die folgende Tabelle gibt den Aufwand der Unterfunktionen an:
        \begin{center}
            \begin{tabular}{cccc}
                \toprule
                \textbf{Funktion} & \texttt{f} & \texttt{g} & \texttt{h} \\
                \midrule
                \textbf{Aufwand} & konstant & linear & konstant \\
                \bottomrule
            \end{tabular}
        \end{center}
    
    \clearpage
    \item
        \begin{myAlgTable}{4}
            \myAlgTableRow{}{\textbf{int} multiplyAllValues(\textbf{ArrayList}<\textbf{Integer}> a,}
            \myAlgTableRow{}{\ \ \ \ \ \ \ \ \ \ \ \ \ \ \ \ \ \ \ \ \ \ \textbf{ArrayList}<\textbf{Integer}> b) \{}
            \myAlgTableRow{1}{\ \ \textbf{int} n = a.size();}
            \myAlgTableRow{1}{\ \ \textbf{int} m = b.size();}
            \myAlgTableRow{1}{\ \ \textbf{if} (a.isEmpty()) \{}
            \myAlgTableRow{}{\ \ \ \ \textbf{return} 0;}
            \myAlgTableRow{}{\ \ \} \textbf{else} \{}
            \myAlgTableRow{n}{\ \ \ \ \textbf{int} aValue = a.remove(0);}
            \myAlgTableRow{}{\ \ \ \ \textbf{int} result = multiplyAllValues(a, b);}
            \myAlgTableRow{m}{\ \ \ \ \textbf{for} (\textbf{int} i = 0; i != m; ++i) \{}
            \myAlgTableRow{1}{\ \ \ \ \ \ \textbf{int} bValue = b.get(i);}
            \myAlgTableRow{1}{\ \ \ \ \ \ result += aValue * bValue;}
            \myAlgTableRow{}{\ \ \ \ \}}
            \myAlgTableRow{}{\ \ \ \ \textbf{return} result;}
            \myAlgTableRow{}{\ \ \}}
            \myAlgTableRow{}{\}}
        \end{myAlgTable}
        
        Für den Aufwand der \texttt{ArrayList}-Methoden gilt folgender Auszug aus
        der \acUrlFootnote{https://docs.oracle.com/en/java/javase/11/docs/api/java.base/java/util/ArrayList.html}{Javadoc}:
        \begin{quote}
            The \texttt{size}, \texttt{isEmpty}, \texttt{get} \textelp{} operations run in constant time.
            \textelp{} All of the other operations run in linear time (roughly speaking).
        \end{quote}
    
    \item
        \begin{myAlgTable}{5}
            \myAlgTableRow{}{\textbf{void} compress(\textbf{LinkedList}<\textbf{Integer}> l) \{}
            \myAlgTableRow{1}{\ \ \textbf{int} n = l.size();}
            \myAlgTableRow{1}{\ \ \textbf{if} (n == 1)}
            \myAlgTableRow{}{\ \ \ \ \textbf{return};}
            \myAlgTableRow{}{}
            \myAlgTableRow{1}{\ \ \textbf{int} element = l.removeFirst();}
            \myAlgTableRow{}{\ \ compress(l);}
            \myAlgTableRow{}{}
            \myAlgTableRow{1}{\ \ \textbf{if} (l.getFirst() != element)}
            \myAlgTableRow{1}{\ \ \ \ l.addFirst(element);}
            \myAlgTableRow{}{\}}
        \end{myAlgTable}
        
        Für den Aufwand der \texttt{LinkedList}-Methoden macht die \acUrlFootnote{https://docs.oracle.com/javase/7/docs/api/java/util/LinkedList.html}{Javadoc} lediglich folgende Aussage:
        \begin{quote}
            All of the operations perform as could be expected for a doubly-linked list.
            Operations that index into the list will traverse the list from the beginning or the end, whichever is closer to the specified index.
        \end{quote}
        Es handelt sich also um eine gewöhnliche doppelte verknüpfte Liste, wobei die \texttt{LinkedList}-Klasse stets eine Referenz auf das aktuell erste und letzte Element der Liste hält.
        Die offensichtliche Optimierung einen Zähler über die aktuelle Anzahl an Elementen zu verwalten, damit dafür nicht jedesmal die ganze Liste traversiert werden muss, ist zusätzlich seit jeher im JDK implementiert.
\end{itemize}
\let\arraystretch\oldarraystretch

\clearpage
Bestimmen Sie die Laufzeit der gebenen Algorithmen mit folgenden Schritten:

\begin{enumerate}
    \item
        \acTask[0.5 Punkte pro Algorithmus]{2.5}:
        Schreiben Sie links in die Platzhalter die \emph{kleinste obere Schranke} für die Laufzeit der jeweiligen Zeile hinein.
        Für \texttt{if}-Abfragen soll der Aufwand der Bedingungen angegeben werden; für Schleifen, die Anzahl der Schleifendurchläufe (ohne Beachtung von äußeren Schleifen).
    
        \acNote:
        Die auszufüllenden Laufzeiten sind nicht immer zwangsweise von \texttt{n} abhängig.
        
        \acNote:
        Geben Sie eine Laufzeit von $\mathcal{O}(\infty)$ an, falls etwas nicht terminiert.
        Falls ein einzelner Teil (bzw. eine Schleife) eines Algorithmus nicht terminiert, sind trotzdem die Laufzeiten für alle anderen Zeilen (auch Zeilen innerhalb der Schleife) anzugeben!
        
        \begin{acSolution}
            Lösung direkt in den jeweiligen Feldern der Algorithmen eingetragen.
        \end{acSolution}
    
    \item\label{subtask:receq}
        \acTask[0.5 Punkte pro Algorithmus]{2.5}:
        Geben Sie die kleinste obere Schranke der Gesamtlaufzeit für jeden Algorithmus in $\mathcal{O}$-Notation an.
        Für die rekursiven Algorithmen genügt es die korrekte Rekursionsgleichung aufzustellen.
        
        \begin{acSolution}
            \begin{itemize}
                \item\textbf{Algorithmus 1}:
                    Der Gesamtaufwand ist $\mathcal{O}(n^2\,\log\,n)$.
                    
                \item\textbf{Algorithmus 2}:
                    Der Gesamtaufwand ist $\mathcal{O}(\infty)$ bzw. der Algorithmus terminiert nicht.
                    
                \item\textbf{Algorithmus 3}:
                    Der Gestamtaufwand ist $\mathcal{O}(n^2)$, weil \texttt{g} folgendermaßen aufgerufen wird:
                    \begin{equation*}
                        \texttt{g(1) g(2) g(4) g(8) \dots\ g(n/4) g(n/2) g(n)}
                    \end{equation*}
                    Da der Aufwand von $g$ linear ist, kann der Aufwand der \texttt{for}-Schleife mit der geometrischen Reihe abgeschätzt werden:
                    \begin{equation*}
                        \mathcal{O}\left(n \sum_{k=0}^\infty \left(\frac{1}{2}\right)^k\right)
                        = \mathcal{O}\left(n\,\frac{1}{1 - \frac{1}{2}}\right)
                        = \mathcal{O}(2\,n)
                        = \mathcal{O}(n)
                    \end{equation*}
                    Ein Gesamtaufwand von $\mathcal{O}(n^2\,\mathrm{log}\,n)$ zählt in dieser Aufgabe als falsch, da dies nicht die kleinste obere Schranke ist.
                    
                \item\textbf{Algorithmus 4}:
                    Für den Gesamtaufwand gilt folgende Rekursionsgleichung:
                    \begin{align*}
                        T(n,m) &= \begin{cases}
                            \mathcal{O}(1) & \text{if $n = 0$} \\
                            \mathcal{O}(n + m) + T(n-1,m) & \text{if $n \neq 0$} \\
                        \end{cases}
                    \end{align*}
                    
                \item\textbf{Algorithmus 5}:
                    Für den Gesamtaufwand gilt folgende Rekursionsgleichung:
                    \begin{align*}
                        T(n) &= \begin{cases}
                            \mathcal{O}(1) & \text{if $n=1$} \\
                            \mathcal{O}(1) + T(n-1) & \text{if $n \neq 1$} \\
                        \end{cases}
                    \end{align*}
            \end{itemize}
        \end{acSolution}
    
    \item
        \acTask[0.5 Punkte pro rekursivem Algorithmus]{1}:
        Lösen Sie die Rekursionsgleichung aus \cref{subtask:receq} zu jeweils einem geschlossenen, nicht-rekursiven Term.
        
        \acNote:
        Für die hier vorkommenden Rekursionsgleichung kann man die Lösung \emph{sehen}.
        Eine Berechnung via Induktion oder Master-Theorem ist nicht nötig.
        
        \begin{acSolution}
            \begin{itemize}
                \item \textbf{Algorithmus 4}: $T(n, m) = \mathcal{O}(n\,(n + m))$.
                \item \textbf{Algorithmus 5}: $T(n) = \mathcal{O}(n)$.
            \end{itemize}
        \end{acSolution}
\end{enumerate}


% ------------------------------------------------------------------------------
\section{Hinweise zur Abgabe}
\label{sec:submission-notes}

Bitte lesen Sie folgende Hinweise zur Abgabe Ihrer Lösungen sorgfältig.
\textbf{Nichtbeachtung führt zum Verlust von 50\% der erlangten Punkte} des jeweiligen Blattes.

\begin{itemize}
    \item Lösungen sind über folgendes SVN-Repository einzureichen:
    
        \url{https://svn.uni-koblenz.de/westteaching/aud-ws1920/<groupname>}
        
        wobei \texttt{<groupname>} durch den Namen Ihrer Abgabegruppe zu ersetzen ist.
    
    \item Lösungen für Übungsblatt \textbf{X} sind im Ordner \texttt{solutions/assignment\textbf{X}} abzulegen.
        Erstellen Sie einen Unterordner \texttt{task\textbf{Y}} für \emph{jede} Aufgabe \textbf{Y} auf dem Übungsblatt.
        Lösungen zu theoretischen Aufgaben werden nur im PDF-Format akzeptiert.
        Für Programmieraufgaben reichen Sie bitte einerseits \texttt{.java}-Dateien ein, hierbei jedoch nur die, die von Ihnen erstellt oder angepasst wurden.
        Reichen Sie bitte jede \texttt{.java}-Datei zusätzlich noch im PDF-Format ein.%
        \footnote{Diese wird während der Korrektur genutzt um einfach Anmerkungen an bestimmten Stellen im Code machen zu können.}
        Erstellen Sie im SVN keine Unterordner pro Java-Package oder vergleichbarem.
        
        Beispielsweise sollte Ihre Verzeichnisstruktur für Aufgabenblatt 0, wobei Aufgabe~1 theoretisch ist und Aufgabe~2 einen Theorie- und einen Programmier-Teil hat, zu dessen Lösung Sie eine Klasse \texttt{Soldier.java} erstellt haben, folgendermaßen aussehen:
        
        \texttt{solutions/\\
        \phantom{xx}assignment0/\\
        \phantom{xxxx}task1/\\
        \phantom{xxxxxx}solution-task1.pdf\\
        \phantom{xxxx}task2/\\
        \phantom{xxxxxx}solution-task2.pdf\\
        \phantom{xxxxxx}Soldier.java\\
        \phantom{xxxxxx}Soldier.pdf}
        
        Verwenden Sie keine Umlaute, Leer- oder Sonderzeichen in Ordner- oder Dateinamen.
    
    \item Jede abgegebene Datei soll dabei Ihren Gruppennamen und die Namen aller, an der Bearbeitung dieses Übungsblatts beteiligten Studenten, aufweisen.
        Nur namentlichen genannte Gruppenmitglieder erhalten Punkte für die jeweilige Aufgabe!
    
    \item Die Korrektur Ihrer Abgaben finden Sie im Ordner \texttt{corrections/}.
    
    \item Den Ordner \texttt{workspace/} können Sie nach Belieben, zur Koordination in Ihrer Gruppe während der Erstellung der Lösung, nutzen.
    
    \item Mit der Abgabe Ihrer Lösung bestätigen Sie, dass Sie das Übungsblatt selbstständig bearbeitet haben, ohne fremde geistige Leistung zu beanspruchen.
        Sollte die Situation eintreten, dass Abgaben verschiedener Gruppen identisch sind, erhalten \emph{alle} beteiligten Gruppen keine Punkte für diese Aufgabe.
    
    \item Es werden nur Programme, die in Java implementiert sind, berücksichtigt.
        Achten Sie darauf, dass Ihre Programme ohne Fehler und Warnung kompiliert werden können.
        Testen Sie dies auf einem Computer, der nicht zu Erstellung Ihrer Lösung verwendet wurde!
        Falls benötigt, dürfen Sie beliebige weitere \texttt{.java}-Dateien erstellen.
        Formattieren Sie Ihren Code lesbar und kommentieren Sie ihn ausreichend.
        Verwenden Sie UTF-8 als Kodierung bei \texttt{.java}-Dateien.
        Falls Testfälle bereitgestellten werden, nutzen Sie diese zur Verifikation Ihrer Lösung.
        Das bestehen alle bereitgestellten Testfälle ist ein \emph{notwendiges} Kriterium zum Erreichen der vollen Punktzahl, kein \emph{hinreichendes}!
    
    \clearpage
    \item Um ein Maven-Projekt in IntelliJ zu importieren:
        
        \begin{enumerate}
            \item \texttt{Import Project} im Projekt-Auswahl-Fenster wählen.
            \item Zu Ordner navigieren, der die \texttt{pom.xml} enthält.
            \item \texttt{Import projects from external model} und \texttt{Maven} auswählen.
            \item Die Standard-Projektkonfiguration mittels \texttt{Next} akzeptieren.
            \item Sicherstellen, dass genau ein Projekt gefunden wurde, und \texttt{Next} auswählen.
            \item Ein JDK mit Version $\geq$ 11 auswählen und mit \texttt{Next} bestätigen.
                (Hier kann es zu Probleme kommen, wenn eine andere als die, in Ihrem Betriebssystem konfigurierte, JDK-Version ausgewählt wird.)
            \item Die Konfiguration mittels \texttt{Finish} abschließen.
        \end{enumerate}
    
        Um nun eine Testklasse auszuführen, Rechtsklick auf die entprechende \texttt{.java}-Datei und \texttt{Run '<class name>'} auswählen.
    
    \item Um ein Maven-Projekt in Eclipse zu importieren:
    
        \begin{enumerate}
            \item \texttt{File} im Fenstermenü auswahlen.
            \item \texttt{Import...} auswählen.
            \item \texttt{Existing Maven Projects} in der Kategorie \texttt{Maven} auswählen.
            \item \texttt{Browse...} auswählen und zum Ordner navigieren, der die \texttt{pom.xml} enthält.
            \item Die Konfiguration mittels \texttt{Finish} abschließen.
        \end{enumerate}
        
        Um nun eine Testklasse auszuführen, Rechtsklick auf die entprechende \texttt{.java}-Datei und erst \texttt{Run As} und dann \texttt{JUnit Test} auswählen.
\end{itemize}

\end{document}